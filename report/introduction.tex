\section{Introduction}
\label{sec:introduction}

The advancement of e-commerce has brought with it a myriad of opportunities and challenges, particularly in the realm of product cataloging and classification. As online marketplaces expand, the task of efficiently categorizing a vast number of products becomes increasingly complex and critical. This challenge is exemplified in the case of Rakuten, a global leader in e-commerce, known for its expansive marketplace that hosts a diverse range of products.

Rakuten, founded in Japan in 1997, revolutionized online shopping with its marketplace concept and has since grown into a major e-commerce platform \cite{brian-2022}. It boasts a community of over 1.3 billion members \cite{statista-2023} and a wide array of services including communications, financial services, and digital content. The Rakuten Institute of Technology (RIT), serving as its research and innovation arm, focuses on areas like computer vision, natural language processing, and human-computer interaction to enhance and innovate within the e-commerce space.

This report addresses a specific challenge posed by Rakuten France: the large-scale multimodal (text and image) classification of products into predefined type codes. In an online marketplace, products are typically accompanied by titles, descriptions, and images. The classification of these products into correct categories is crucial for various aspects of e-commerce, such as personalized recommendations, search optimization, and efficient query processing. However, the task is not straightforward due to the sheer volume of products, the variety of classes, and the common issue of unbalanced data distributions in large catalogs.

The primary objective of this challenge is to develop a model capable of accurately categorizing products based on their textual and visual information. This involves predicting the appropriate product type code for each item in Rakuten France's catalog, a task that requires an intricate understanding of both the textual and visual characteristics of the products. The complexity of this challenge is heightened by the intrinsic variability and potential inconsistencies in product labels and images.

To facilitate this endeavor, Rakuten France has provided a dataset comprising approximately 99,000 product listings in a CSV format, which includes both training and test sets. The dataset contains product titles, detailed descriptions, images, and corresponding product type codes. The benchmark for this challenge is the weighted-F1 score, a metric that balances precision and recall, and is particularly useful in scenarios with uneven class distributions \cite{10.1093/mnrasl/slac120}.

In summary, this report delves into the development of a multimodal classification system for Rakuten's extensive product catalog. The focus is on leveraging both textual and visual data to achieve accurate and efficient product categorization, a task essential for enhancing the user experience and operational efficiency in the dynamic world of e-commerce.

%\subsection{Background}
%\label{subsec:background}
%Sentence here. % Template text for the Background subsection.

%\subsection{Motivation}
%\label{subsec:motivation}
%Sentence here. % Template text for the Motivation subsection.

%\subsection{Aim and Objectives}
%\label{subsec:aims}
%Sentence here. % Template text for the Aim and Objectives subsection.

%\subsection{Scope}
%\label{subsec:scope}
%Sentence here. % Template text for the Scope subsection.

%\subsection{Document Structure}
%\label{subsec:structure}
%Sentence here. % Template text for the Document Structure subsection.